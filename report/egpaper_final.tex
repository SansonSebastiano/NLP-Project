\documentclass[10pt,twocolumn,letterpaper]{article}

\usepackage{cvpr}
\usepackage{times}
\usepackage{epsfig}
\usepackage{graphicx}
\usepackage{amsmath}
\usepackage{amssymb}

% Include other packages here, before hyperref.

% If you comment hyperref and then uncomment it, you should delete
% egpaper.aux before re-running latex.  (Or just hit 'q' on the first latex
% run, let it finish, and you should be clear).
\usepackage[breaklinks=true,bookmarks=false]{hyperref}

\cvprfinalcopy % *** Uncomment this line for the final submission

\def\cvprPaperID{****} % *** Enter the CVPR Paper ID here
\def\httilde{\mbox{\tt\raisebox{-.5ex}{\symbol{126}}}}

% Pages are numbered in submission mode, and unnumbered in camera-ready
%\ifcvprfinal\pagestyle{empty}\fi
\setcounter{page}{4321}
\begin{document}

%%%%%%%%% TITLE
\title{WhatsApp chat summarization}

\author{Andrea Auletta\\
{\tt\small andrea.auletta@studenti.unipd.it}
% For a paper whose authors are all at the same institution,
% omit the following lines up until the closing ``}''.
% Additional authors and addresses can be added with ``\and'',
% just like the second author.
% To save space, use either the email address or home page, not both
\and
Davide Baggio\\
{\tt\small davide.baggio@studenti.unipd.it}
\and
Marco Bernardi\\
{\tt\small marco.bernardi.11@studenti.unipd.it}
\and
Marco Brigo\\
{\tt\small marco.brigo@studenti.unipd.it}
\and
Sebastiano Sanson\\
{\tt\small sebastiano.sanson@studenti.unipd.it}
}

\maketitle
%\thispagestyle{empty}

%%%%%%%%% ABSTRACT
\begin{abstract}This research project in Natural Language Processing (NLP) 
   focuses on the development of an automated system for the summarization 
   of group chat messages. The objective is to generate concise summaries of conversations, 
   thereby obviating the need for users to review all messages individually. 
   This application is especially pertinent in contexts where chat logs proliferate quickly, 
   offering significant benefits in terms of time savings and cognitive load reduction for users.
\end{abstract}

%%%%%%%%% BODY TEXT
\section{Introduction}

Please follow the guidelines that have been uploaded on Moodle. 

\section{Dataset}

\subsection{SAMSum}

The SAMSum~\cite{DBLP:journals/corr/abs-1911-12237} dataset contains about 16k messenger-like conversations with summaries. 
Conversations were created and written down by linguists fluent in English. 
Linguists were asked to create conversations similar to those they write on a daily basis, reflecting the proportion of topics of their real-life messenger convesations. 
The style and register are diversified - conversations could be informal, semi-formal or formal, 
they may contain slang words, emoticons and typos. Then, the conversations were annotated with summaries. 
It was assumed that summaries should be a concise brief of what people talked about in the conversation in third person. 
The SAMSum dataset was prepared by Samsung R\&D Institute Poland and is distributed for research purposes

\subsection{Dialogsum}

DialogSum~\cite{chen-etal-2021-dialogsum} is an extensive dialogue summarization dataset comprising 13,460 dialogues, supplemented by an additional 100 holdout dialogues designated for topic generation. 
Each dialogue is paired with manually annotated summaries and topics.

The dataset is exclusively in English and includes four data fields: the text of the dialogue, a human-written summary of the dialogue, a human-written topic or one-liner of the dialogue, and a unique identifier for each example. 
The data splits are as follows: 12,460 dialogues for training, 500 dialogues for validation, 1,500 dialogues for testing, and 100 holdout dialogues containing only the id, dialogue, and topic fields.

DialogSum distinguishes itself from previous datasets by incorporating dialogues under rich real-life scenarios, including a wider array of task-oriented contexts. 
The dialogues exhibit clear communication patterns and intents, making them suitable for summarization.

\subsection{Custom Dataset}



\section{Models Involved}

\section{Fine Tuning}

\section{Testing}

\section{Results}

\subsection{Language}

All manuscripts must be in English.

%\subsection{Suggested Structure}

%The following is a suggested structure for your report:

%\begin{itemize}
%	\item Introduction (20\%): describe the problem you are working on, why it's important, what are your goals, and provide also an overview of your main results.
%	\item Dataset (20\%): describe the data you are working with for your project. What type of data is it? Where did it come from? How much data are you working with? Did you have to do any preprocessing, filtering, etc., and why?
%	\item Method (30\%): discuss your approach for solving the problems that you set up in the introduction. Why is your approach the right thing to do? Did you consider alternative approaches? It may be helpful to include figures, diagrams, or tables to describe your method or compare it with others.
%	\item Experiments (30\%): discuss the experiments that you performed. The exact experiments will vary depending on the project, but you might compare with prior work, perform an ablation study to determine the impact of various components of your system, experiment with different hyperparameters or architectural choices. You should include graphs, tables, or other figures to illustrate your experimental results.
%\end{itemize}	

%-------------------------------------------------------------------------
{\small
\bibliographystyle{ieee_fullname}
\bibliography{egbib}
}

\end{document}
