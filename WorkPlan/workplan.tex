\documentclass[12pt]{article}
\usepackage{amsmath,amssymb}
\usepackage{color}
\usepackage{enumitem}
\usepackage{hyperref}
\usepackage{listings}
\usepackage{graphicx}

\lstset{
  basicstyle=\ttfamily,
  mathescape
}

% MAKE TITLE AND AUTHOR
\title{\textbf{Project workplan} \\ 
    \large Text summarizer for chat messages in a group chat}

\author{
    Andrea Auletta
    \and
    Davide Baggio
    \and
    Marco Brigo
    \and
    Marco Bernardi
    \and
    Sebastiano Sanson
}

\date{\today}
\makeindex
\begin{document}
\maketitle
\newpage
\tableofcontents
\newpage


\section{Short summary}
Our Natural Language Processing (NLP) project aims to automate the summarization of group 
chat messages, providing users with concise summaries of conversations without the need to 
read through all messages. Imagine a scenario where you're part of bustling team chat or a 
lively online community. Messages fly back and forth very quickly about various topics and 
discussions. Obviously chat logs accumulate rapidly and it might happen that you miss some 
important points, requiring significant time and effort to read through the entirety of the
convesation history. Our aim is to save users time and cognifitive effort when facing 
situations like this. By leveraging state-of-the-art natural language processing techniques 
our system will analyze the content of group chat messages, summarizing the most relevant 
information, the key points, main topics, and notable highlights of the discussion, 
presenting users with a condensed version of the conversation that is easy to digest and 
comprehend. \\ 
The idea is to try several existing approaches that can be useful to reach an optimal goal. 
Then we will try to create our own version by using the techniques that we found to be the 
most effective.
\section{Tasks}
\begin{center}
    \begin{tabular}{ |p{1.5cm}|p{3cm}|p{2cm}|p{3cm}|} 
        \hline
        Number & Task & Mandatory & Participants\\
        \hline
        1 & Study of the state-of-the-art of the topic & Yes & Auletta, Baggio, Bernardi, Brigo, Sanson\\ 
        \hline
        2 &  Dataset collection & Yes & Auletta, Baggio, Bernardi \\
        \hline
        3 & Data preprocessing & Yes & Brigo, Sanson \\
        \hline
        4 & Summarization of text messages & Yes & Auletta, Baggio, Bernardi, Brigo, Sanson \\
        \hline
        5 & Summarization of images & No & Auletta, Brigo, Sanson \\
        \hline
        6 & Summarization of audio & No & Baggio, Bernardi \\
        \hline
        7 & Test and evaluation of the model & Yes & Auletta, Baggio, Bernardi, Brigo, Sanson \\
        \hline
        8 & Documentation & Yes & Auletta, Baggio, Bernardi, Brigo, Sanson \\
        \hline     
    \end{tabular}
\end{center}
\newpage

\section{Deliverables}
\begin{enumerate}
    \item \textbf{Study of the state-of-the-art of the topic}:
    \begin{itemize}
        \item \textbf{Deliverable}: Report of the state-of-the-art;
        \item \textbf{Description of the deliverable}: Each member of the group will be assigned a certain number of papers that 
        they will have to read and summarize. At the end of this process, a final report condensing all the 
        information will be produced, showcasing the state-of-the-art in the field;
        \item \textbf{Measurable objectives}: We will report about at least 15 papers in our state of the art 
        analysis (3 for each of us);
    \end{itemize}
    \item \textbf{Dataset collection}:
    \begin{itemize}
        \item \textbf{Deliverable}: Dataset;
        \item \textbf{Description of the deliverable}: A dataset of chat messages will be collected or found;
    \end{itemize}
    \item \textbf{Data preprocessing}:
    \begin{itemize}
        \item \textbf{Deliverable}: Preprocessed and analyzed dataset;
        \item \textbf{Description of the deliverable}: The dataset will be preprocessed and analyzed to be 
        used in the model. A notebook file will be produced (.ipynb);
    \end{itemize}
    \item \textbf{Summarization of text messages}:
    \begin{itemize}
        \item \textbf{Deliverable}: Model (text messages);
        \item \textbf{Description of the deliverable}: An implementation of the model that can summarize text 
        messages will be produced;
    \end{itemize}
    \item \textbf{Summarization of images}:
    \begin{itemize}
        \item \textbf{Deliverable}: Model (images);
        \item \textbf{Description of the deliverable}: An implementation of the model that can translate images in text and 
        then summarize them with the text;
    \end{itemize}
    \item \textbf{Summarization of audio}:
    \begin{itemize}
        \item \textbf{Deliverable}: Model (audio);
        \item \textbf{Description of the deliverable}: An implementation of the model that can write down the audio and  
        then summarize them with the text;
    \end{itemize}
    \item \textbf{Test and evaluation of the model}:
    \begin{itemize}
        \item \textbf{Deliverable}: Evaluation report;
        \item \textbf{Description of the deliverable}: The model will be tested and evaluated;
        \item \textbf{Measurable objectives}: The model will be evaluated in different ways:
        \begin{itemize}
            \item Manual evaluation: We will check the readibility of the summaries, so if the produced summary does not contain gaps in its rhetorical structure or dangling anaphora;
            \item Automatic evaluation: we will use some metrics to evaluate the model, for example ROUGE-1, ROUGE-2 and F-Measure. We will try to achieve a ROUGE $>$ 0.4, and F-Measure $>$ 0.5;
        \end{itemize}
    \end{itemize}
    \item \textbf{Documentation}:
    \begin{itemize}
        \item \textbf{Deliverable}: Documentation;
        \item \textbf{Description of the deliverable}: The necessary documentation will be produced to understand the project;
    \end{itemize} 
\end{enumerate}

\end{document}