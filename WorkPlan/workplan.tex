\documentclass[12pt]{article}
\usepackage{amsmath,amssymb}
\usepackage{color}
\usepackage{enumitem}
\usepackage{hyperref}
\usepackage{listings}
\usepackage{graphicx}

\lstset{
  basicstyle=\ttfamily,
  mathescape
}

% MAKE TITLE AND AUTHOR
\title{\textbf{Project workplan} \\ 
    \large Text summarizer for chat messages in a group chat}

\author{
    Andrea Auletta
    \and
    Davide Baggio
    \and
    Marco Brigo
    \and
    Marco Bernardi
    \and
    Sebastiano Sanson
}

\date{\today}
\makeindex
\begin{document}
\maketitle
\newpage
\tableofcontents
\newpage


\section{Short summary}
Our Natural Language Processing (NLP) project aims to automate the summarization of group 
chat messages, providing users with concise summaries of conversations without the need to 
read through all messages. Imagine a scenario where you're part of bustling team chat or a 
lively online community. Messages fly back and forth very quickly about various topics and 
discussions. Obviously chat logs accumulate rapidly and it might happen that you miss some 
important points, requiring significant time and effort to read through the entirety of the
convesation history. Our aim is to save users time and cognifitive effort when facing 
situations like this. By leveraging state-of-the-art natural language processing techniques 
our system will analyze the content of group chat messages, summarizing the most relevant 
information, the key points, main topics, and notable highlights of the discussion, 
presenting users with a condensed version of the conversation that is easy to digest and 
comprehend. \\ 
The idea is to try several existing approaches that can be useful to reach an optimal goal. 
Then we will try to create our own version by using the techniques that we found to be the 
most effective.
\section{Tasks}
\begin{center}
    \begin{tabular}{ |p{1.5cm}|p{3cm}|p{2cm}|p{3cm}|} 
        \hline
        Number & Task & Mandatory & Participants\\
        \hline
        1 & Study of the state-of-the-art of the topic & Yes & Auletta, Baggio, Bernardi, Brigo, Sanson\\ 
        \hline
        2 &  Dataset collection & Yes & Auletta, Baggio, Bernardi \\
        \hline
        3 & Data preprocessing & Yes & Brigo, Sanson \\
        \hline
        4 & Summarization of text messages & Yes & Auletta, Baggio, Bernardi, Brigo, Sanson \\
        \hline
        5 & Summarization of images & No & Auletta, Brigo, Sanson \\
        \hline
        6 & Summarization of audio & No & Baggio, Bernardi \\
        \hline
        7 & Test and evaluation of the model & Yes & Auletta, Baggio, Bernardi, Brigo, Sanson \\
        \hline
        8 & Documentation & Yes & Auletta, Baggio, Bernardi, Brigo, Sanson \\
        \hline     
    \end{tabular}
\end{center}
\newpage
\section{Deliverables}
\begin{center}
    \begin{tabular}{ |p{1.5cm}|p{3cm}|p{6cm}|p{3cm}|} 
        \hline
        Task & Deliverable & Description & Measurable objectives\\
        \hline
        1 & Report of the state-of-the-art & Each member of the group will be assigned a certain number of papers that they will have to read and summarize. At the end of this process, a final report condensing all the information will be produced, showcasing the state-of-the-art in the field. & We will report about at least 15 papers in our state of the art analysis (3 for each of us) \\
        \hline
        2 & Dataset & A dataset of chat messages will be collected & Collect a minimum number of chats, the number will not be small but will be chosen depending on the model that will be developed \\
        \hline
        3 & Preprocessed and analyzed dataset & The dataset will be preprocessed and analyzed to be used in the model. A notebook file will be produced &\\
        \hline
        4 & Model (text messages) & An implementation of the model that can summarize text messages will be produced &\\
        \hline
        5 & Model (images) & The model will be able to understand images and will summirize their content & \\
        \hline
        6 & Model (audio) & The model will be able to understand audio and will summirize their content &\\
        \hline
        7 & Evaluation report & The model will be tested and evaluated, and a report will be produced &\\
        \hline
        8 & Documentation & Will be produced all the documents which are useful to the comprehension of the project & \\
        \hline
    \end{tabular}
\end{center}
\section{Metrics}




\end{document}